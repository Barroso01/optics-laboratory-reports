\label{sec:INTRO}
Quantum electrodynamics provides the most fundamental description of light up to date. However, the classical formulation developed in the early years of this field yields a very elegant, intuitive and simple understanding of light. Furthermore, the mathematical models built around this classical theory are tailor made for the level of the physical experimentation discussed in this work. From this classical standpoint, light can be considered to be a harmonic solution (wave) to the second order, linear and partial differential equation derived from Maxwell's equations \cite{maxwell1954electricity}. Because of this considerations, we expect for light to exhibit wave like properties. \\

One of such properties is interference. By interfering two light beams, it is possible for the emerging beam to sum up constructively, destructively or some state in the middle (Figure \ref{fig:BasicInterf}). The parameter that primarily governs interference is the phase of the wave. However, it is important to keep track of the amplitudes when constructing optical experiments because in order for interference to occur a series of conditions must hold.  In the case of the electromagnetic waves, the conditions for interference are that the light source must be monochromatic, coherent and polarized in the same orientation. \\

This work aims to explain the basic experimental considerations that need surveillance when working with interferometers. This is done by designing and implementing a type of amplitude splitting interferometer known as Mach-Zhender. Later, a series of measurements and modifications are implemented to characterize the phenomena known as interference.



