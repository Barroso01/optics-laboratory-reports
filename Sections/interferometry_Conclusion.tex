Throughout this work, we exposed the principal elements to consider when dealing with interferometers. The practice successfully exposes the interaction of optical interference by making optical arrangements with the principal experimental elements like retarders, polarizers, cameras and LASER beams. Furthermore, the key mathematical foundations are mentioned in order to aboard a complete view on the results obtained in experiment. \\

The results of the experiments presented where very satisfactory because they illustrate what are the main aspects to consider when working with interferometers. The team concludes that it is important to control the collimation of the beams in order to control the number of interference patterns. It is also critical to monitor which elements introduce a change in the optical path because this also can affect the orientation, shape and contrast of the interference. Polarization is another important characteristic to control in the laboratory. Failing to do so can lead to unsuccessful attempts of generating interference. 

Finally it is important to select the right interferometer for the experiment being done since every configurations has its own practical advantages.The applications of the interferometers discussed in this work generalize to the following statements 1) Michelson interferometers can have measurement applications with huge resolution values (in the scale of nanometers). 2) Mach Zehnder applications for research because of the wide options for interfering light with different characteristics. 3) Sagnac interferometer is not vulnerable to environmental noise and therefore is ideal for practical applications with few specifics on interference. 
